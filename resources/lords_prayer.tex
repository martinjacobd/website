\documentclass[12pt]{article}
\usepackage[T1]{fontenc}
\usepackage{hyperref}
\usepackage{fontspec}

\author{Jacob Martin}
\date{\today}
\title{The Lord's Prayer}

\setmainfont{Times New Roman}

\begin{document}

\maketitle

\section{Introduction}

We are told that the Word of the Lord never returns void.  Often, we can return to the same passage, and find something that we hadn't before considered.  I am going to go through the Lord's Prayer phrase-by-phrase and see what gems lie.  We begin in Matthew 6, verse 9.
\begin{quote}
After this manner therefore pray ye: \\
Our Father which art in heaven, Hallowed be Thy name. \\
Thy Kingdom come \\
Thy will be done in earth, as it is in heaven. \\
Give us this day our daily bread. \\
And forgive us our debts, as we forgive our debtors. \\
And lead us not into temptation, but deliver us from evil: \\
For Thine is the kingdom, and the power, \\
and the glory, for ever. Amen.
\end{quote}

\section{After this manner therefore pray ye}
\label{sec-2}
Jesus begins us in a lesson.  That He should have to \emph{tell} persons how to pray seems strange.  Most of us grew up hearing, if not also saying, prayers.  Luke's version of the Lord's Prayer even begins in a petition of one of His followers who said, ``Lord, teach us to pray, as John also taught his disciples.''  (Luke 11:1) These people were yearning after God.  They were yearning to speak to Him and know Him.  They were yearning for a communion---as we all yearn.  People often speak of the ``God-shaped hole'' that humans have.  I prefer to think of it as an inborn desire to commune with our Creator.  We had this communion in the Garden of Eden, and original sin revoked it.  Since, the human heart has striven to reunite itself with its creator.  We were all still created in the image of God, and our soul desperately yearns to return to this image and cease defiling it.

The sermon on the mount, in which the Lord's Prayer is contained here, was preached in an area of Israel.  There were no mountains in Galilee proper, where Jesus had been teaching, but the most probable location of the sermon was in some large hills to the west of the Sea of Galilee.  In other words, He was preaching to Jews.  It is probable that many of these Jews had been hellenized and were not well-acquianted with the rich Jewish prayer traditions.  Others, however, would have been familiar with these.  They still, however, petitioned for a way of prayer.  All were aware that the existing communion was not enough.

Just two verses earlier, he warned us not to pray in vain repetitions.  This is not to criticize rote prayer \emph{per se.}  The word here connotes more of the ``babbling'' in which the Greeks participated, similar to speaking in unknown tongues without translation.  In fact, the ESV translates this phrase as ``empty phrases.''  Another indication we have that this does not criticize rote prayer is that the Pharisees had many rote prayer traditions, and Jesus did not here include a criticism of them.  Anyways, these are all indications Jesus was teaching these people how to pray---how to commune with the Father in the Holy Spirit.

Mind you, this was very necessary.  Pentecost had not yet come.  The Holy Spirit had not descended on the people.  The ordinary manner of prayer in which we now pray, ``To the Father, through the Holy Spirit, In the name of the Son'' would have been entirely alien to these first-century persons.  Jesus filled a wide need when he taught these people to pray.

\section{Our Father which art in heaven}

In the first meat of the prayer, we find that prayer is indeed a confrontation with the almighty.  Moses spoke to Jesus face-to-face; Adam and Even walked with the Lord in the Garden.  We commune with the Father---indeed all of the God-head---in prayer.  Of course, when considering a text, we must always consider the speaker.  Here, Jesus, our Lord, is speaking---Jesus, the only begotten.  He, however, speaks for all of us, God's children.  But we do see the first confrontation, or communion, we have in prayer, namely that of our Father, in \emph{Heaven.}

\section{Hallowed be thy name}

Here, in this communion, we kneel before the Lord our God and reverence him---even up to his \emph{name.}  We are beckoned to bow before the Lord in our hearts---in the closet of our hearts---in utter humility.  In awe that even the \emph{name} of the Lord has power and reverence.  A name to first-century believers would have carried a persons whole personality and reputation.  Jesus declares this in the subjunctive mood---a grammatical feature almost absent in today's English.  This mood carried the idea that he was requesting an action that should be done.  In it was almost an exhortation to those there---an exhortation to keep the name of the Lord unblemished from scandal and sin.

This also emphasizes the \emph{sacramental} nature of prayer.  But in prayer, we engage in an almost sacramental work---a visual, physical exercise in which we effect God's real grace.  We do a sort of command here.  By just saying the name ought to be more holy, we glorify it, working out of our own free will to effect a real grace.  (Of course, more formally and pedantically, prayer isn't a \emph{sacrament} as it was instituted long before Christ's earthly ministry, and it doesn't work \emph{ex opere operato}.)

We must always notice the primary effect of prayer as being in our own souls.  It stretches our desire for God. So, here, in recognizing the reverence due Him and His Name, we stretch our desire for Him, kindling the flame of the soul---viz., the yearnings towards and groanings from the Holy Spirit.

\section{Thy Kingdom come}
\label{sec-5}
Here we see the next kind of communion in which we participate in prayer---viz., communion with the other believers, the rest of God's Kingdom.  Again, Jesus is making this request in the subjunctive.  In modern English we may form this as, ``Let your kingdom come,'' or ``May your kingdom come.''  He is requesting that God bring his Kingdom, which evokes both ideas of the eventual reign the Lord's Kingdom will have over the whole earth and of the reign the Church universal extends now.  We engage in communion with other believers in prayer, and we also engage in prayer when in communion with other believers.  For instance, we pray as a congregation at church, both in spoken prayer and by praising our Lord in hymns.  I would even argue that the preaching is a form of prayer: We exalt the Lord and his word and speak of him and with him, for He is everwhere.  Again, this also reflects the quasi-sacramental nature of prayer in that we are asking for a grace.  Hebrews even echoes that we are participating in a long line of prayer, prayer even from past believers.  (Hebrews 12:1)

It also seems strange that we should command God, but this is the ordinary way of prayer, as set out even in the Old Testament by Isaiah 45:11.
\begin{quote}
Thus saith the Lord, the Holy One of Israel, \\
and his Maker, Ask me of things to come concerning \\
my sons, and concerning the work of my hands command ye me.
\end{quote}

We should also note what a terrifying thing we are asking for when we ask for God's kingdom.  The Lord's very presence casts out all evil.  We are asking also for judgment when we ask for His kingdom, and the path to His kingdom is narrow. Without having received the Body of Christ and had faith in Him, without having put our hands to the plow and not looking back, without having scorned the world, we will not inherit the Kingdom of Heaven, and the outer darkness is dark indeed.

\section{Thy will be done on earth, as it is in heaven}

This phrase furthers our humility and the primacy of God that we underscored in our discussion of the holiness of the Lord's name.  Again, in prayer we exalt the Lord---all of him, including his desires.  Also, this is also an exhortation to us believers to do the Lord's will, just as James, the Lord's brother, exhorted us to ``be doers of the word and not hearers only.''  This request is to the Lord, but we also are the Lord's servants, and the actions of His servant's are the Lord's ordinary way of effecting his will here on earth.

The phrase ``On earth as it is in Heaven'' also recalls that we are constantly engaged in a spiritual realm.  We are often tempted to think of spiritual life as a subset of our life on earth.  We think, we have work, school, home-life, social-life, and then prayer or spiritual life.  In reality, all those other things are a subset of our spiritual life.  The so-called spiritual world, with God and the Angels preexisted the earth and from it, the earth was created.  Man was created in God's image, not the other way around as we often like to think.  In some sense, even if our petitions in prayer do not get effected immediately here on earth, prayer is the most real, present activity in which men can engage.  Prayer is the way in which we commune with the spiritual world which is around and above our world, not inside of it.  We are tempted to compartmentalize our spirituality into a box and set it there and go about our lifestyles.  In reality, we ought to compartmentalize our physical life into a box as we proceed spiritually.

Again, we also notice the effect of prayer in our own soul. We increase our yearnings for God and kindle the flame of the Spirit by acknowledging God's sovereignty.

\section{Give us this day our daily bread}

This also echoes that the spiritual world is the ``real'' world.  We petition God, for our even our very base needs in this world.  Jesus was poor and homeless, a mendicant who depended on others and God explicitly.  It is often echoed that many middle-class Americans do not know what this phrase truly means to the impoverished.  I posit that we know exactly what it means.  Sure, few of us wonder where our daily bread will come from.  But who among us is not acquianted with the terror that comes with uncertainty about our livelihoods---be it concern over our jobs, our houses, or our parents and children.  Today, we depend every bit as much on God as the first-century Christians did.  This also shows us that God owns all things, and that we must ask for everything, for it was never ours to begin with.  Also, this underscores an honesty in prayer.  We are to go to the Father; it is not wrong to want or to desire---as we are often tempted to think---but rather we go to God in honesty about our desires.  (Cf. Jesus's prayer before being crucified.)

\section{Forgive us our debts, as we forgive our debtors}

He we enter into the third type of communion in prayer.  First, we commune with the God-head.  Second, with the believers in the kingdom of Heaven.  Now, we see communion with all persons, even our enemies.  The Lord instructed us to love our enemies, and He speaks much of forgiveness.  If we have a gift at the altar and remember a quarrel with another, He tells us earlier in the Sermon on the Mount, we are to go and try to reconcile ourselves immediately.  He warns us in the parable of the debtors in Matthew 26, that we are not to pursue those who have wronged us, for we have been given a far greater redemption.  We now answer to a higher authority.  He even says if someone sues us for our shirt, then we are to give our jacket also (my version).  This both echoes God's power as supreme judge and redemptor, and exhorts us to show the same forgiveness in everyday life.  In the verses immediately following the prayer, we are warned that if we do not forgive others, we will not be forgiven.

\section{Lead us not into temptation, but deliver us from evil}

Of course, God would never lead us into temptation, but we do pray for deliverence from evil.  Evil here may be translated as ``the evil one,'' viz., Satan.  The idea here is that we request power to resist these temptations when the evil one comes.  We also, here, request deliverence for the evil we ourselves have made.  Again, this beckons us back into the idea of a spiritual realm---a realm over which God exerts total control and which is very real.  We pray for deliverence from the forces of the enemy in the battle we fight and the race we run.

\section{For Thine is the kingdom and the power and the glory forever.  Amen.}

This final benediction is only included in some manuscripts.  In many other church traditions, it is not included in the formulas they use in the liturgy and in recitation.  Most protestant churches today, however, include it as a regular part of the prayer, perhaps because of its inclusion in the King James Bible.  This part of the prayer, even if it isn't part of the original, is still a strong benediction and declaration of the Lord's supremacy and power.  All things belong to Him, even the most intangible, powerful, and substantive things extant belong to Him---His kingdom, His power, His glory.  This echoes our communion with Him.  We bow before His supremacy and power to realize that He has control over all things.  All things are rightfully His.

\section{Conclusion}

The Lord shows us to be humble and adoring when we go to Him in prayer.  In prayer, we commune not only with the Lord, but also with other believers and even non-believers.  The Lord's prayer shows us the sheer power and supremacy the Lord has over all things and the power of our communion with Him through prayer. This prayer shows explicitly how we express desire towards and increasing yearning for Our Most Holy Lord.

\end{document}
